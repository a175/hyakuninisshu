%\documentclass[a4paper,12pt,draft]{amsart}
\documentclass[a4paper,12pt]{jsarticle}
\usepackage{amsmath, amsthm, amssymb}
\usepackage{url}
\usepackage{braket}

\theoremstyle{plain}
\newtheorem{theoremX}{Theorem}

\newtheorem{thm}{Theorem}[section]
\newtheorem{theorem}[thm]{Theorem}
\newtheorem{question}[thm]{Question}
\newtheorem{lemma}[thm]{Lemma}
\newtheorem{cor}[thm]{Corollary}
\newtheorem{proposition}[thm]{Propsition}

\newtheorem{conjecture}[thm]{Conjecture}

\theoremstyle{remark}  %%amsart
\newtheorem{remark}[thm]{Remark}
\newtheorem{rem}[thm]{Remark}
\newtheorem{example}[thm]{Example}
\newtheorem*{ackn}{Acknowledgments}

\theoremstyle{definition}  %%amsart
\newtheorem{definition}[thm]{Definition}
\newtheorem{algorithm}[thm]{Algorithm}
\newtheorem{problem}[thm]{Problem}


\newcommand{\ZZ}{\mathbb{Z}}
\newcommand{\CC}{\mathbb{C}}
\newcommand{\KK}{\mathbb{K}}
\newcommand{\NN}{\mathbb{N}}
\newcommand{\RR}{\mathbb{R}}
\newcommand{\QQ}{\mathbb{Q}}
\newcommand{\AAA}{\mathcal{A}}
\newcommand{\BBB}{\mathcal{B}}
\newcommand{\DDD}{\mathcal{D}}
\newcommand{\HHH}{\mathcal{H}}
\newcommand{\LLL}{\mathcal{L}}
\newcommand{\III}{\mathcal{I}}
\newcommand{\CCC}{\mathcal{C}}
\newcommand{\NNN}{\mathcal{N}}
\newcommand{\FFF}{\mathcal{F}}
\newcommand{\MMM}{\mathcal{M}}
\newcommand{\PPP}{\mathcal{P}}
\newcommand{\QQQ}{\mathcal{Q}}
\newcommand{\RRR}{\mathcal{R}}
\newcommand{\SSS}{\mathcal{S}}
\newcommand{\TTT}{\mathcal{T}}
\newcommand{\EEE}{\mathcal{E}}



\begin{document}
\cite{wiki:数学者の一覧}による年代の分類:
\section{初期の有名な数学者}
\subsection{古代ギリシア・ローマの有名な数学者}
\subsubsection{タレス}
タレス\cite{wiki:タレス}
(紀元前624年 - 紀元前546年頃、古代ギリシアの哲学者)

\begin{theoremX}[ターレスの定理]
円周上の2つの点を結ぶ線分が円の中心を含むとする.
このとき,
2点と円周上の別の点とを結ぶ2つの線分のなす角は必ず直角である.
\end{theoremX}

\begin{remark}
古代ギリシャの哲学者、数学者タレスにちなんで名付けられた。
その前にもこの定理は発見されていたが、
タレスが初めてピラミッドの高さを発見した事からこの名前が生まれた。
\cite{wiki:タレスの定理}
\end{remark}


\subsubsection{ピタゴラス}
ピタゴラス\cite{wiki:ピタゴラス}
(紀元前582年 - 紀元前496年、ギリシャ).

\begin{theoremX}[ピタゴラスの定理]
  直角三角形の斜辺の長さを$a$, 他の2辺の長さをそれぞれ$b$, $c$とする.
  このとき, $a^2=b^2+c^2$.
\end{theoremX}

\begin{remark}
「ピタゴラスが直角二等辺三角形のタイルが敷き詰められた床を見ていて、
  この定理を思いついた」などいくつかの逸話が伝えられているが、
実際にこの定理にピタゴラス自身が関わった事があるかから全く分かっていない。
\cite{wiki:ピタゴラスの定理}
\end{remark}

\subsubsection{エウクレイデス}
エウクレイデス
\cite{wiki:エウクレイデス}
(紀元前365年? - 紀元前275年?、アレクサンドリア): 別名ユークリッド 幾何学原論、素数の一意性.

\begin{theoremX}
  素数全体は有限個と仮定し, 全ての素数の積をNとする.
  このとき$N+1$はどの素数でも割り切れないがどの素数よりも真に大きい.
\end{theoremX}

\subsubsection{アルキメデス}
アルキメデス(紀元前287年 - 紀元前212年、シチリア): 求積法、極限操作、アルキメデスの原理

\subsubsection{エラトステネス}
    エラトステネス(紀元前275年 - 紀元前194年、ヘレニズム時代のエジプトで活躍したギリシャ人): エラトステネスの篩

4世紀生まれの有名な数学者

5世紀生まれの有名な数学者

6世紀生まれの有名な数学者

8世紀生まれの有名な数学者

11世紀生まれの有名な数学者

12世紀生まれの有名な数学者

14世紀生まれの有名な数学者

15世紀生まれの有名な数学者

16世紀生まれの有名な数学者

\section{17世紀生まれの有名な数学者}

\section{18世紀生まれの有名な数学者}

\section{19世紀生まれの有名な数学者}
1801年 -- 1810年生まれの有名な数学者

1811年 -- 1820年生まれの有名な数学者

1821年 -- 1830年生まれの有名な数学者

1831年 -- 1840年生まれの有名な数学者

1841年 -- 1850年生まれの有名な数学者

1851年 -- 1860年生まれの有名な数学者

1861年 -- 1870年生まれの有名な数学者

1871年 -- 1880年生まれの有名な数学者

1881年 -- 1890年生まれの有名な数学者

1891年 -- 1900年生まれの有名な数学者

\section{20世紀生まれの有名な数学者}
1901年 -- 1910年生まれの有名な数学者

1911年 -- 1920年生まれの有名な数学者

1921年 -- 1930年生まれの有名な数学者

1931年 -- 1940年生まれの有名な数学者

1941年 -- 1950年生まれの有名な数学者

1951年 -- 1960年生まれの有名な数学者

1961年 -- 1970年生まれの有名な数学者

1971年 -- 1980年生まれの有名な数学者


\bibliography{wikipedia}
\bibliographystyle{jplain}

\end{document}
