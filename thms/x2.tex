%\documentclass[a4paper,12pt,draft]{amsart}
\documentclass[a4paper,12pt]{jsarticle}
\usepackage{amsmath, amsthm, amssymb}
\usepackage{url}
\usepackage{braket}

\theoremstyle{plain}
\newtheorem{theoremX}{Theorem}

\newtheorem{thm}{Theorem}[section]
\newtheorem{theorem}[thm]{Theorem}
\newtheorem{question}[thm]{Question}
\newtheorem{lemma}[thm]{Lemma}
\newtheorem{cor}[thm]{Corollary}
\newtheorem{proposition}[thm]{Propsition}

\newtheorem{conjecture}[thm]{Conjecture}

\theoremstyle{remark}  %%amsart
\newtheorem{remark}[thm]{Remark}
\newtheorem{rem}[thm]{Remark}
\newtheorem{example}[thm]{Example}
\newtheorem*{ackn}{Acknowledgments}

\theoremstyle{definition}  %%amsart
\newtheorem{definition}[thm]{Definition}
\newtheorem{algorithm}[thm]{Algorithm}
\newtheorem{problem}[thm]{Problem}


\newcommand{\ZZ}{\mathbb{Z}}
\newcommand{\CC}{\mathbb{C}}
\newcommand{\KK}{\mathbb{K}}
\newcommand{\NN}{\mathbb{N}}
\newcommand{\RR}{\mathbb{R}}
\newcommand{\QQ}{\mathbb{Q}}
\newcommand{\AAA}{\mathcal{A}}
\newcommand{\BBB}{\mathcal{B}}
\newcommand{\DDD}{\mathcal{D}}
\newcommand{\HHH}{\mathcal{H}}
\newcommand{\LLL}{\mathcal{L}}
\newcommand{\III}{\mathcal{I}}
\newcommand{\CCC}{\mathcal{C}}
\newcommand{\NNN}{\mathcal{N}}
\newcommand{\FFF}{\mathcal{F}}
\newcommand{\MMM}{\mathcal{M}}
\newcommand{\PPP}{\mathcal{P}}
\newcommand{\QQQ}{\mathcal{Q}}
\newcommand{\RRR}{\mathcal{R}}
\newcommand{\SSS}{\mathcal{S}}
\newcommand{\TTT}{\mathcal{T}}
\newcommand{\EEE}{\mathcal{E}}



\begin{document}
\cite{wiki:数学者の一覧}による年代の分類:
\section{初期の有名な数学者}
\subsection{古代ギリシア・ローマの有名な数学者}
\subsubsection{タレス}
タレス\cite{wiki:タレス}
(紀元前624年 - 紀元前546年頃、古代ギリシアの哲学者)

\begin{theoremX}[ターレスの定理]
円周上の2つの点を結ぶ線分が円の中心を含むとする.
このとき,
2点と円周上の別の点とを結ぶ2つの線分のなす角は必ず直角である.
\end{theoremX}

\begin{remark}
古代ギリシャの哲学者、数学者タレスにちなんで名付けられた。
その前にもこの定理は発見されていたが、
タレスが初めてピラミッドの高さを発見した事からこの名前が生まれた。
\cite{wiki:タレスの定理}
\end{remark}


\subsubsection{ピタゴラス}
ピタゴラス\cite{wiki:ピタゴラス}
(紀元前582年 - 紀元前496年、ギリシャ).

\begin{theoremX}[ピタゴラスの定理]
  直角三角形の斜辺の長さを$a$, 他の2辺の長さをそれぞれ$b$, $c$とする.
  このとき, $a^2=b^2+c^2$.
\end{theoremX}

\begin{remark}
「ピタゴラスが直角二等辺三角形のタイルが敷き詰められた床を見ていて、
  この定理を思いついた」などいくつかの逸話が伝えられているが、
実際にこの定理にピタゴラス自身が関わった事があるかから全く分かっていない。
\cite{wiki:ピタゴラスの定理}
\end{remark}

\subsubsection{エウクレイデス}
エウクレイデス
\cite{wiki:エウクレイデス}
(紀元前365年? - 紀元前275年?、アレクサンドリア): 別名ユークリッド 幾何学原論、素数の一意性.

\begin{theoremX}
  素数全体は有限個と仮定し, 全ての素数の積をNとする.
  このとき$N+1$はどの素数でも割り切れないがどの素数よりも真に大きい.
\end{theoremX}



\subsubsection{アルキメデス}
アルキメデス(紀元前287年 - 紀元前212年、シチリア): 求積法、極限操作、アルキメデスの原理

\subsubsection{エラトステネス}
    エラトステネス(紀元前275年 - 紀元前194年、ヘレニズム時代のエジプトで活躍したギリシャ人): エラトステネスの篩

\subsection{4世紀生まれの有名な数学者}
\subsubsection{ヒュパティア}
ヒュパティア
\cite{wiki:ヒュパティア}
(370頃-415、ギリシア):数学の教育や著述をした人として知られる最初の女性    

\subsection{5世紀生まれの有名な数学者}
\subsubsection{アールヤバタ}
アールヤバタ
\cite{wiki:アーリヤバタ}
(476 - 550-、インド):代数学、無限小、微分方程式、線型方程式の解


\subsection{6世紀生まれの有名な数学者}
\subsubsection{ブラーマグプタ}
ブラーマグプタ
\cite{wiki:ブラフマグプタ}
(598-668、インド):0と他の整数との加減乗除、ブラーマグプタの公式、ブラーマグプタの二平方恒等式
\begin{theoremX}
  円に内接する四角形の各辺の長さを
  時計回りに$a$, $b$, $c$, $d$とし, $s$を半周長とし,
  四角形の面積を$S$とする.
  このとき, $S=\sqrt{(s-a)(s-b)(s-c)(s-d)}$
\end{theoremX}
\begin{remark}
  ヘロンの公式の一般化
\end{remark}
\begin{remark}
  ブラーマグプタ自身は円に内接するという条件を明示していないため、不正確な公式としてのみ記録に残っている。
  \cite{wiki:ブラーマグプタの公式}
\end{remark}

\subsection{8世紀生まれの有名な数学者}
\subsubsection{フワーリズミー}
フワーリズミー
\cite{wiki:フワーリズミー}
  (780-850、イラク):最古の代数学書を著述、インドの記数法を紹介[8]

\subsection{11世紀生まれの有名な数学者}
\subsubsection{ウマル・ハイヤーム}
ウマル・ハイヤーム
\cite{wiki:ウマル・ハイヤーム}
(1048-1131、イラン):3次方程式の解法、二項展開の発見[9]  
  

\subsection{12世紀生まれの有名な数学者}
\subsubsection{バースカラ2世}
バースカラ2世
\cite{wiki:バースカラ2世}
(1114-1185、インド):2次方程式、3次方程式、4次方程式の解法、解析学

\subsubsection{レオナルド・フィボナッチ}
レオナルド・フィボナッチ
\cite{wiki:レオナルド・フィボナッチ}
(1179年頃-1250年頃、イタリア):フィボナッチ数列.

\subsection{14世紀生まれの有名な数学者}
\subsubsection{マーダヴァ}
マーダヴァ
\cite{wiki:マーダヴァ}
(1340-1425、インド):無限級数の展開、ケーララ学派の創始者

\begin{theoremX}
  $a_n= \frac{(-1)^n}{2n+1}$とし, $\pi$を円周率とする.
  このとき,
  交代級数$\sum_{n=0}^{\infty}a_n$は$\frac{\pi}{4}$に収束する.
\end{theoremX}
\begin{remark}
  この公式を名付けたのはライプニッツであるが.
  これはすでに15世紀のインドの数学者マーダヴァがライプニッツより
  300年ほど前に発見していたものである.
  公式の発見がマーダヴァの功績であることを示すために
  マーダヴァ-ライプニッツ級数と呼ばれることもある.
  \cite{wiki:ライプニッツの公式}
\end{remark}

\subsection{15世紀生まれの有名な数学者}
\subsubsection{ルカ・パチョーリ}
ルカ・パチョーリ
\cite{wiki:ルカ・パチョーリ}
(1445-1517、イタリア):「会計の父」
     
\subsection{16世紀生まれの有名な数学者}
\subsubsection{ジェロラモ・カルダーノ}
ジェロラモ・カルダーノ
\cite{wiki:ジェロラモ・カルダーノ}
(1501-1578、イタリア): 3次方程式の解法、虚数概念の導入

\subsubsection{ルドヴィコ・フェラーリ}
ルドヴィコ・フェラーリ
\cite{wiki:ルドヴィコ・フェラーリ}
1522-1565、イタリア):4次方程式の解法

\subsubsection{クリストファー・クラヴィウス}
クリストファー・クラヴィウス
\cite{wiki:クリストファー・クラヴィウス}
(1538-1612、ドイツ/イタリア):グレゴリオ暦の編纂

\subsubsection{ジョン・ネイピア}
ジョン・ネイピア
\cite{wiki:ジョン・ネイピア}
(1550-1617、スコットランド):対数の発見

\subsubsection{パウル・ギュルダン}
パウル・ギュルダン
\cite{wiki:パウル・ギュルダン}
     (1577-1643、スイス/オーストリア):パップス=ギュルダンの定理
\begin{theoremX}
  平面上にある有界な曲線$C$の長さを$s$とし,
  $C$と同じ平面上にあり$C$と共有点を持たない軸$l$の周りで
  $C$を一回転させた回転面の面積を$S$とする.
  また, 回転させる曲線$C$の重心$G$から回転軸$l$までの距離を$R$とする.
  このとき,
  $S = 2\pi Rs$.
\end{theoremX}
\begin{remark}
  アレキサンドリアのパップスによって4世紀に発見され,
  後にパウル・ギュルダンによって独立に発見された.
  \cite{wiki:パップス=ギュルダンの定理}
\end{remark}

\subsubsection{マラン・メルセンヌ}
\cite{wiki:マラン・メルセンヌ}
     (1588-1648、フランス):ヨーロッパ中の学者たちとの交流、メルセンヌ数
\subsection{ジラール・デザルグ}
ジラール・デザルグ
\cite{wiki:ジラール・デザルグ}
     (1591-1661、フランス):射影幾何学の基礎、デザルグの定理     
\begin{theoremX}
  三角形$ABC$と$abc$は同一平面上にはないとし,
  直線$Aa$と$Bb$と$Cc$が一点で交わるとする.
  直線$AB$と$ab$、$BC$と$bc$, $CA$と$ca$の交点を, それぞれ$X$, $Y$, $Z$とする.
  このとき, $X$, $Y$, $Z$は同一平面上に存在する.
\end{theoremX}
\cite{wiki:デザルグの定理}

\subsubsection{ルネ・デカルト}
ルネ・デカルト
\cite{wiki:ルネ・デカルト}
(1596-1650、フランス):デカルト座標系、解析幾何学の祖

\begin{theoremX}
  球と位相同型な多面体の不足角の総和を$D$とする.
  このとき, $D=4\pi$.
\end{theoremX}
\cite{wiki:不足角}

\section{17世紀生まれの有名な数学者}
\subsection{ピエール・ド・フェルマー}
ピエール・ド・フェルマー
\cite{wiki:ピエール・ド・フェルマー}
(1607?-1665、フランス):フェルマーの小定理、フェルマーの最終定理
\begin{theoremX}
  $p$を素数とし,
  $a$を整数とする.
  このとき,
  $a^p \equiv a \pmod{p}$.
\end{theoremX}
\cite{wiki:フェルマーの小定理}

\subsubsection{ブレーズ・パスカル}
ブレーズ・パスカル
\cite{wiki:ブレーズ・パスカル}
(1623-1662、フランス):パスカルの定理、確率論の創始
\begin{theoremX}
  六角形$ABCDEF$は円に内接するとする.
  直線$AB$と$DE$との交点を$P$,
  直線$BC$と$EF$との交点を$Q$,
  直線$CD$と$FA$との交点を$R$とおく.
  このとき,
  $P$, $Q$, $R$は同一直線上にある.
\end{theoremX}
\cite{wiki:パスカルの定理}

\subsection{アイザック・ニュートン}
アイザック・ニュートン
\cite{wiki:アイザック・ニュートン}
(1642-1727、イギリス):二項定理、微分積分学
\begin{theoremX}
実数$r$と非負整数$k$に対し,
  $\binom{r}{k}=\frac{r(r-1)\cdots(r-k+1)}{k!}$
とし,
$|x|>|y|$とする.
このとき,
$(x-y)^r=\sum_{k=0}^{\infty} x^{r-k}y^k$,
\end{theoremX}
\cite{wiki:二項定理}

\section{関孝和}
関孝和
\ciite{wiki:関孝和}
(1642-1708、日本):和算家、行列式、ベルヌーイ数の発見

\section{ゴットフリート・ライプニッツ}
ゴットフリート・ライプニッツ
ゴットフリート・ライプニッツ
(1646-1716、ドイツ):微分積分学、行列式、形式言語

\begin{theorem}
  $u,v$を微分可能な関数とする.
  このとき,
  $\frac{d}{dx}(u\cdot v) = u\cdot \frac{dv}{dx}  + \frac{du}{dx} \cdots v$.
\end{theorem}
\cite{wiki:積の微分法則}
  
    ヤコブ・ベルヌーイ(1654-1705、スイス):ベルヌーイ数
    アブラーム・ド・モアブル(1667-1754、フランス):ド・モアブルの定理
    ヨハン・ベルヌーイ(1667-1748、スイス):ヤコブ・ベルヌーイの弟、カテナリー曲線
    ブルック・テイラー (1685-1731、イギリス): テイラー展開
    コリン・マクローリン (1698-1746、スコットランド): マクローリン展開
    ダニエル・ベルヌーイ(1700-1782、スイス):ヨハン・ベルヌーイの子、ベルヌーイの定理
\section{18世紀生まれの有名な数学者}

\section{19世紀生まれの有名な数学者}
1801年 -- 1810年生まれの有名な数学者

1811年 -- 1820年生まれの有名な数学者

1821年 -- 1830年生まれの有名な数学者

1831年 -- 1840年生まれの有名な数学者

1841年 -- 1850年生まれの有名な数学者

1851年 -- 1860年生まれの有名な数学者

1861年 -- 1870年生まれの有名な数学者

1871年 -- 1880年生まれの有名な数学者

1881年 -- 1890年生まれの有名な数学者

1891年 -- 1900年生まれの有名な数学者

\section{20世紀生まれの有名な数学者}
1901年 -- 1910年生まれの有名な数学者

1911年 -- 1920年生まれの有名な数学者

1921年 -- 1930年生まれの有名な数学者

1931年 -- 1940年生まれの有名な数学者

1941年 -- 1950年生まれの有名な数学者

1951年 -- 1960年生まれの有名な数学者

1961年 -- 1970年生まれの有名な数学者

1971年 -- 1980年生まれの有名な数学者


\bibliography{wikipedia}
\bibliographystyle{jplain}

\end{document}
